\documentclass[a4paper,fleqn,usenatbib]{mnras}
%DIF LATEXDIFF DIFFERENCE FILE
%DIF DEL paper_old.tex   Sat Sep 15 12:39:51 2018
%DIF ADD paper.tex       Sat Sep 15 12:35:04 2018
%=========================================================================
\usepackage{amsmath}
\usepackage{amssymb}
\usepackage{graphicx}
\usepackage{grffile}
\usepackage{float}
\usepackage[dvips]{epsfig}
\usepackage{epsfig}
\usepackage{dblfloatfix}
\usepackage{color}
\usepackage{caption}
\usepackage{hyperref}
\usepackage{bm}
\usepackage[british]{babel}
%Non reposionated tables
\newcommand{\HI}{{\text{H\MakeUppercase{\romannumeral 1}}} }
\newcommand{\lya}{\ifmmode{{\rm Ly}\alpha}\else Ly$\alpha$\ \fi}
\newcommand{\kms}{\ifmmode\mathrm{km\ s}^{-1}\else km s$^{-1}$\fi}
\newcommand{\vrot}{\ifmmode v_{\mathrm{rot}}\else $v_{\mathrm{rot}}$~\fi}
\newcommand{\vout}{\ifmmode v_{\mathrm{out}}\else $v_{\mathrm{out}}$~\fi}
\newcommand{\tauh}{\ifmmode \tau_{\mathrm{H}}\else $\tau_{\mathrm{H}}$~\fi}
\newcommand{\vth}{\ifmmode v_{\mathrm{th}}\else $v_{\mathrm{th}}$~\fi}
\newcommand{\hatk}{\ifmmode \hat{k}\else $\hat{k}$~\fi}
\newcommand{\STD}{\ifmmode \mathrm{STD}\else $\mathrm{STD}$~\fi}
\newcommand{\SKW}{\ifmmode \mathrm{SKW}\else $\mathrm{SKW}$~\fi}
\newcommand{\BI}{\ifmmode \mathrm{BI}\else $\mathrm{BI}$~\fi}
%DIF PREAMBLE EXTENSION ADDED BY LATEXDIFF
%DIF UNDERLINE PREAMBLE %DIF PREAMBLE
\RequirePackage[normalem]{ulem} %DIF PREAMBLE
\RequirePackage{color}\definecolor{RED}{rgb}{1,0,0}\definecolor{BLUE}{rgb}{0,0,1} %DIF PREAMBLE
\providecommand{\DIFaddtex}[1]{{\protect\color{blue}\uwave{#1}}} %DIF PREAMBLE
\providecommand{\DIFdeltex}[1]{{\protect\color{red}\sout{#1}}}                      %DIF PREAMBLE
%DIF SAFE PREAMBLE %DIF PREAMBLE
\providecommand{\DIFaddbegin}{} %DIF PREAMBLE
\providecommand{\DIFaddend}{} %DIF PREAMBLE
\providecommand{\DIFdelbegin}{} %DIF PREAMBLE
\providecommand{\DIFdelend}{} %DIF PREAMBLE
%DIF FLOATSAFE PREAMBLE %DIF PREAMBLE
\providecommand{\DIFaddFL}[1]{\DIFadd{#1}} %DIF PREAMBLE
\providecommand{\DIFdelFL}[1]{\DIFdel{#1}} %DIF PREAMBLE
\providecommand{\DIFaddbeginFL}{} %DIF PREAMBLE
\providecommand{\DIFaddendFL}{} %DIF PREAMBLE
\providecommand{\DIFdelbeginFL}{} %DIF PREAMBLE
\providecommand{\DIFdelendFL}{} %DIF PREAMBLE
%DIF HYPERREF PREAMBLE %DIF PREAMBLE
\providecommand{\DIFadd}[1]{\texorpdfstring{\DIFaddtex{#1}}{#1}} %DIF PREAMBLE
\providecommand{\DIFdel}[1]{\texorpdfstring{\DIFdeltex{#1}}{}} %DIF PREAMBLE
\newcommand{\DIFscaledelfig}{0.5}
%DIF HIGHLIGHTGRAPHICS PREAMBLE %DIF PREAMBLE
\RequirePackage{settobox} %DIF PREAMBLE
\RequirePackage{letltxmacro} %DIF PREAMBLE
\newsavebox{\DIFdelgraphicsbox} %DIF PREAMBLE
\newlength{\DIFdelgraphicswidth} %DIF PREAMBLE
\newlength{\DIFdelgraphicsheight} %DIF PREAMBLE
% store original definition of \includegraphics %DIF PREAMBLE
\LetLtxMacro{\DIFOincludegraphics}{\includegraphics} %DIF PREAMBLE
\newcommand{\DIFaddincludegraphics}[2][]{{\color{blue}\fbox{\DIFOincludegraphics[#1]{#2}}}} %DIF PREAMBLE
\newcommand{\DIFdelincludegraphics}[2][]{% %DIF PREAMBLE
\sbox{\DIFdelgraphicsbox}{\DIFOincludegraphics[#1]{#2}}% %DIF PREAMBLE
\settoboxwidth{\DIFdelgraphicswidth}{\DIFdelgraphicsbox} %DIF PREAMBLE
\settoboxtotalheight{\DIFdelgraphicsheight}{\DIFdelgraphicsbox} %DIF PREAMBLE
\scalebox{\DIFscaledelfig}{% %DIF PREAMBLE
\parbox[b]{\DIFdelgraphicswidth}{\usebox{\DIFdelgraphicsbox}\\[-\baselineskip] \rule{\DIFdelgraphicswidth}{0em}}\llap{\resizebox{\DIFdelgraphicswidth}{\DIFdelgraphicsheight}{% %DIF PREAMBLE
\setlength{\unitlength}{\DIFdelgraphicswidth}% %DIF PREAMBLE
\begin{picture}(1,1)% %DIF PREAMBLE
\thicklines\linethickness{2pt} %DIF PREAMBLE
{\color[rgb]{1,0,0}\put(0,0){\framebox(1,1){}}}% %DIF PREAMBLE
{\color[rgb]{1,0,0}\put(0,0){\line( 1,1){1}}}% %DIF PREAMBLE
{\color[rgb]{1,0,0}\put(0,1){\line(1,-1){1}}}% %DIF PREAMBLE
\end{picture}% %DIF PREAMBLE
}\hspace*{3pt}}} %DIF PREAMBLE
} %DIF PREAMBLE
\LetLtxMacro{\DIFOaddbegin}{\DIFaddbegin} %DIF PREAMBLE
\LetLtxMacro{\DIFOaddend}{\DIFaddend} %DIF PREAMBLE
\LetLtxMacro{\DIFOdelbegin}{\DIFdelbegin} %DIF PREAMBLE
\LetLtxMacro{\DIFOdelend}{\DIFdelend} %DIF PREAMBLE
\DeclareRobustCommand{\DIFaddbegin}{\DIFOaddbegin \let\includegraphics\DIFaddincludegraphics} %DIF PREAMBLE
\DeclareRobustCommand{\DIFaddend}{\DIFOaddend \let\includegraphics\DIFOincludegraphics} %DIF PREAMBLE
\DeclareRobustCommand{\DIFdelbegin}{\DIFOdelbegin \let\includegraphics\DIFdelincludegraphics} %DIF PREAMBLE
\DeclareRobustCommand{\DIFdelend}{\DIFOaddend \let\includegraphics\DIFOincludegraphics} %DIF PREAMBLE
\LetLtxMacro{\DIFOaddbeginFL}{\DIFaddbeginFL} %DIF PREAMBLE
\LetLtxMacro{\DIFOaddendFL}{\DIFaddendFL} %DIF PREAMBLE
\LetLtxMacro{\DIFOdelbeginFL}{\DIFdelbeginFL} %DIF PREAMBLE
\LetLtxMacro{\DIFOdelendFL}{\DIFdelendFL} %DIF PREAMBLE
\DeclareRobustCommand{\DIFaddbeginFL}{\DIFOaddbeginFL \let\includegraphics\DIFaddincludegraphics} %DIF PREAMBLE
\DeclareRobustCommand{\DIFaddendFL}{\DIFOaddendFL \let\includegraphics\DIFOincludegraphics} %DIF PREAMBLE
\DeclareRobustCommand{\DIFdelbeginFL}{\DIFOdelbeginFL \let\includegraphics\DIFdelincludegraphics} %DIF PREAMBLE
\DeclareRobustCommand{\DIFdelendFL}{\DIFOaddendFL \let\includegraphics\DIFOincludegraphics} %DIF PREAMBLE
%DIF END PREAMBLE EXTENSION ADDED BY LATEXDIFF

\begin{document}

%=========================================================================
%		FRONT MATTER
%=========================================================================
\DIFdelbegin %DIFDELCMD < \title[Lyman-α photons through rotating outflows]{%%%
\DIFdelend \DIFaddbegin \title[Lyman-$\alpha$ photons through rotating
  outflows]{\DIFaddend Lyman-$\alpha$ photons through rotating outflows}  
\author[M.C. Remolina-Gutierrez \& J.E. Forero-Romero]{
  Maria Camila Remolina-Guti\'errez$^{1}$
  \thanks{mc.remolina197@uniandes.edu.co} \&
  Jaime E. Forero-Romero $^{1}$
  \thanks{je.forero@uniandes.edu.co}\\
  %%
  $^{1}$ Departamento de F\'isica, Universidad de los Andes, Cra. 1
  No. 18A-10 Edificio Ip, CP 111711, Bogot\'a, Colombia \\
}

\maketitle

\begin{abstract}
Outflows and rotation are two ubiquitous kinematic features in the gas
kinematics of galaxies. 
Here we \DIFdelbegin \DIFdel{perform }\DIFdelend \DIFaddbegin \DIFadd{introduce a semi-analytic model to quantify how rotating
outflows impact the morphology of the Lyman-$\alpha$ emission line.   
The model is contrasted against }\DIFaddend Monte Carlo radiative transfer
simulations of outflowing gas with additional solid body rotation\DIFdelbegin \DIFdel{to understand how these kinematic
features impact the morphology of the Lyman-$\alpha$ emission line}\DIFdelend .
We explore a range of neutral Hydrogen optical depth of
$10^5\leq\tauh\leq10^7$, rotational velocity $0\leq \vrot/\kms \leq
100$ and outflow velocity $0\leq \vout/\kms\leq 50$.  
We find three \DIFdelbegin \DIFdel{important }\DIFdelend consequences of rotation.
First, it introduces a dependency with viewing angle; second it
\DIFdelbegin \DIFdel{produces a line broadening }\DIFdelend \DIFaddbegin \DIFadd{broadens the line }\DIFaddend and third it increases the flux at the
line's center.
\DIFdelbegin \DIFdel{We also test a }\DIFdelend \DIFaddbegin \DIFadd{For all viewing angles, the }\DIFaddend semi-analytic model \DIFdelbegin \DIFdel{that modifies the
spectra of
pure outflow simulations and manages to reproduce the }\DIFdelend \DIFaddbegin \DIFadd{reproduces the
}\DIFaddend radiative transfer results for the line \DIFdelbegin \DIFdel{widths }\DIFdelend \DIFaddbegin \DIFadd{width }\DIFaddend and flux change at the
line's center within a $7\%$ and $50\%$ precision \DIFdelbegin \DIFdel{, respectively, }\DIFdelend for an optical depth of
$\tauh=10^5$\DIFaddbegin \DIFadd{, respectively, }\DIFaddend and within $2\%$ and $1\%$ for an optical
depth of $\tauh=10^7$.
\DIFdelbegin \DIFdel{We argue that rotation is a required kinematic feature to
take into
account for a correct interpretation of }\DIFdelend \DIFaddbegin \DIFadd{Using this model we also show that the peaks of integrated spectra
taken from opposite sides of an edge-on rotating gas distribution
should have a separation of $\frac{1}{2}\vrot$. 
The semi-analytic model presented here is a convenient tool to
introduce rotational kinematics as a post-processing step of idealized
Monte Carlo simulations; it provides a framework to interpret }\DIFaddend \lya
spectra in systems where rotation is \DIFdelbegin \DIFdel{already evident }\DIFdelend \DIFaddbegin \DIFadd{expected or directly measured
}\DIFaddend through kinematic maps\DIFdelbegin \DIFdel{but also in other
systems, such as Green Pea galaxies, where a strong }%DIFDELCMD < \lya %%%
\DIFdel{emission at
the line's center could be explained by rotational kinematics}\DIFdelend .  
\end{abstract}

\begin{keywords}
galaxies:ISM --- line:profiles --- radiative transfer --- methods: numerical
\end{keywords}


%=========================================================================
%		PAPER CONTENT
%=========================================================================

%*************************************************************************

\section{Introduction}
\label{sec:intro}

Recent advances in instrumentation have revealed the presence of gas
rotation on vastly different physical scales.
For instance, spatially resolved spectra on compact dwarf galaxies
have measured clear signs of gas showing pure rotation kinematics
\citep{2015A&A...577A..21C,2017A&A...600A.125C} and the recent mapping of
high redshift circumgalactic regions have also revealed kinematic
evidence for large scale rotation \citep{2018MNRAS.473.3907A}.
\DIFdelbegin \DIFdel{These systems }\DIFdelend \DIFaddbegin \DIFadd{Systems }\DIFaddend with star formation, neutral gas and low dust
contents \DIFdelbegin \DIFdel{naturally show }\DIFdelend \DIFaddbegin \DIFadd{can produce }\DIFaddend a \lya emission line \citep{PartridgePeebles}
motivating the observational work to phenomenologically link \DIFaddbegin \DIFadd{tracers
of galaxy rotation such as }\DIFaddend H$\alpha$ \DIFdelbegin \DIFdel{kinematics }\DIFdelend to \lya spectra
\citep[e.g.][]{Herenz2016}.  

What is then the expected imprint of rotation on a resonant emission
line such as the \lya line? To what extent is it possible to constrain
rotational kinematics from the \lya emission line? 
Detailed radiative transfer (RT) \lya modeling of rotating systems
started until recently by \cite{Garavito14}.
In that work the authors studied the influence of pure solid body
rotation on the \lya line's morphology.
They found that rotation indeed introduces changes, the most
noticeable being the dependence of the spectra with the viewing
angle with respect to the rotation axis. 

\cite{Garavito14} also presented a simple semi-analytical
approximation that accounted for the main features of the \lya spectra
from a rotation sphere.
Recently, this semi-analytic solution was used to perform a Markov
Chain Monte Carlo exploration to fit the observed spectra Compact
Dwarf Galaxy \citep{tololo} with atypical features that could be
explained by pure rotation.  

However, the gas dynamics in Lyman Alpha Emitter (LAE) galaxies are
more complex than pure rotation.
\DIFdelbegin \DIFdel{The case for outflows as traced in the }%DIFDELCMD < \lya %%%
\DIFdel{line is much better
established.
}\DIFdelend In many observations the \lya line profile has a single peak
redwards from the line's center, in other cases there is a double peak
but the peak on the red side is stronger
\citep[e.g.][]{2010ApJ...717..289S,Erb14,Trainor16}.   
These features have been \DIFdelbegin \DIFdel{readily }\DIFdelend explained as the consequence of multiple
\lya photon scatterings through a homogeneous outflowing shell of
neutral Hydrogen
\DIFdelbegin \DIFdel{\mbox{%DIFAUXCMD
\citep{2006A&A...460..397V,Orsi12,2012ApJ...751...29Y,2015ApJ...812..123G}}%DIFAUXCMD
}\DIFdelend \DIFaddbegin \DIFadd{\mbox{%DIFAUXCMD
\citep{Verhamme06,Orsi12,2015ApJ...812..123G}}%DIFAUXCMD
}\DIFaddend .  

\DIFdelbegin \DIFdel{An }\DIFdelend \DIFaddbegin \DIFadd{Nevertheless, a }\DIFaddend study of the combined effects of outflows and
rotation has not been presented in the literature.
Here we \DIFdelbegin \DIFdel{present }\DIFdelend \DIFaddbegin \DIFadd{report on }\DIFaddend such a study \DIFaddbegin \DIFadd{with the main aim of quantifying the
validity of the semi-analytic approximation presented by
\mbox{%DIFAUXCMD
\cite{Garavito14} }%DIFAUXCMD
in the case where outflows are also present}\DIFaddend . 
We investigate a simplified geometrical configuration corresponding to a
spherical gas cloud with symmetrical radial outflows and solid body
rotation \DIFdelbegin \DIFdel{. 
We base our modeling on }\DIFdelend \DIFaddbegin \DIFadd{and contrast the semi-analytic model against the results of }\DIFaddend a
Monte-Carlo radiative transfer code. 
\DIFdelbegin \DIFdel{Besides modeling the impact of joint rotation and outflows, we also
check to what extent the analytical model presented by
\mbox{%DIFAUXCMD
\cite{Garavito14} }%DIFAUXCMD
to explain the effects of rotation can also be
applied in our case.
}\DIFdelend 

The structure of the paper is the following.
We introduce first our theoretical tools and assumptions
in Section \ref{sec:theory}. We continue in Section \ref{sec:results}
with the results from the Monte-Carlo simulation, the comparison
against the semi-analytical approximation which we use to make a
thorough exploration of the effect of rotation.
In Section \ref{sec:discussion} we discuss our results and their
possible implications for observational analysis to finally present
our conclusions in Section \ref{sec:conclusions}.

\DIFaddbegin \DIFadd{Throughout the paper we use a thermal velocity for a neutral Hydrogen
gas of }\vth \DIFadd{$= 12.86$ }\kms\DIFadd{, which corresponds to a temperature of
$T=10^4$ K. 
}

\DIFaddend \begin{figure*}
\centering
    \includegraphics[width=0.48\textwidth]{doppler_shift_logtau6_theta90}
    \includegraphics[width=0.48\textwidth]{doppler_shift_logtau6_theta0}
  \caption{\textbf{Qualitative trends of changing outflow and
      rotational velocity viewed perpendicular/paralell to the
      rotation axis}.  
    Here we fix $\tauh=10^6$. 
    The six panels on the left correspond to $\theta=90^\circ$ and the
    panels on the right to $\theta=0^{\circ}$
    We vary \vrot increasing from left to right and \vout increasing
    from top to bottom. 
    The thin black line corresponds to the \lya line obtained with
    CLARA without any rotation and the indicated outflow velocity.
    The thick black line corresponds to the results including both
    outflows and rotation.
    The thick gray line shows the results of modifying the pure outflow
    solution by the Doppler shift presented in Equation \ref{eq:shift_x}
    (in thin line), if there is a radiative transfer of rotation and outflows
    (thick and clear line), and if there is a radiative transfer of
    only outflows, but also a Doppler shift from the rotational
    velocity (thick and dark line).  
    \label{fig:doppler_shift}}
\end{figure*}

\begin{figure*}
\begin{center}
\includegraphics[height=0.25\textwidth]{line_characterization_std_vout5_logtau5.pdf}
\includegraphics[height=0.25\textwidth]{line_characterization_std_vout25_logtau5.pdf}
\includegraphics[height=0.25\textwidth]{line_characterization_std_vout50_logtau5.pdf}\\
\includegraphics[height=0.25\textwidth]{line_characterization_std_vout5_logtau6.pdf}
\includegraphics[height=0.25\textwidth]{line_characterization_std_vout25_logtau6.pdf}
\includegraphics[height=0.25\textwidth]{line_characterization_std_vout50_logtau6.pdf}\\
\includegraphics[height=0.25\textwidth]{line_characterization_std_vout5_logtau7.pdf}
\includegraphics[height=0.25\textwidth]{line_characterization_std_vout25_logtau7.pdf}
\includegraphics[height=0.25\textwidth]{line_characterization_std_vout50_logtau7.pdf}
\end{center}
\caption{\textbf{Standard Deviation trends.} Results for all the
  Radiative Transfer simulations (in triangles) compares against the
  Doppler Shift model (lines).
  All panels correspond to a viewing angle of $\theta = 90^{\circ}$
  (perpendicular to the rotation axis). 
  The optical depth increases from top to bottom and the outflow
  velocity from left to right.
  \label{fig:standard_deviation}}
\end{figure*}

\begin{figure*}
\begin{center}
\includegraphics[height=0.25\textwidth]{line_characterization_skw_vout5_logtau5}
\includegraphics[height=0.25\textwidth]{line_characterization_skw_vout25_logtau5}
\includegraphics[height=0.25\textwidth]{line_characterization_skw_vout50_logtau5}\\
\includegraphics[height=0.25\textwidth]{line_characterization_skw_vout5_logtau6}
\includegraphics[height=0.25\textwidth]{line_characterization_skw_vout25_logtau6}
\includegraphics[height=0.25\textwidth]{line_characterization_skw_vout50_logtau6}\\
\includegraphics[height=0.25\textwidth]{line_characterization_skw_vout5_logtau7}
\includegraphics[height=0.25\textwidth]{line_characterization_skw_vout25_logtau7}
\includegraphics[height=0.25\textwidth]{line_characterization_skw_vout50_logtau7}
\end{center}
\caption{\textbf{Skewness trends.} Results for all the
  Radiative Transfer simulations (in triangles) compares against the
  Doppler Shift model (lines).
  Follows the same layout as Figure \ref{fig:standard_deviation}. 
  \label{fig:skewness}}
\end{figure*}

\begin{figure*}
\begin{center}
\includegraphics[height=0.25\textwidth]{line_characterization_bi_vout5_logtau5}
\includegraphics[height=0.25\textwidth]{line_characterization_bi_vout25_logtau5}
\includegraphics[height=0.25\textwidth]{line_characterization_bi_vout50_logtau5}\\
\includegraphics[height=0.25\textwidth]{line_characterization_bi_vout5_logtau6}
\includegraphics[height=0.25\textwidth]{line_characterization_bi_vout25_logtau6}
\includegraphics[height=0.25\textwidth]{line_characterization_bi_vout50_logtau6}\\
\includegraphics[height=0.25\textwidth]{line_characterization_bi_vout5_logtau7}
\includegraphics[height=0.25\textwidth]{line_characterization_bi_vout25_logtau7}
\includegraphics[height=0.25\textwidth]{line_characterization_bi_vout50_logtau7}
\end{center}
\caption{\textbf{Bimodality trends.} Results for all the
  Radiative Transfer simulations (in triangles) compares against the
  Doppler Shift model (lines). 
  Follows the same layout as Figure \ref{fig:standard_deviation}. 
  \label{fig:bimodality}}
\end{figure*}

\begin{figure*}
\begin{center}
\includegraphics[height=0.25\textwidth]{line_characterization_vi_vout5_vrot100_logtau5}
\includegraphics[height=0.25\textwidth]{line_characterization_vi_vout25_vrot100_logtau5}
\includegraphics[height=0.25\textwidth]{line_characterization_vi_vout50_vrot100_logtau5}\\
\includegraphics[height=0.25\textwidth]{line_characterization_vi_vout5_vrot100_logtau6}
\includegraphics[height=0.25\textwidth]{line_characterization_vi_vout25_vrot100_logtau6}
\includegraphics[height=0.25\textwidth]{line_characterization_vi_vout50_vrot100_logtau6}\\
\includegraphics[height=0.25\textwidth]{line_characterization_vi_vout5_vrot100_logtau7}
\includegraphics[height=0.25\textwidth]{line_characterization_vi_vout25_vrot100_logtau7}
\includegraphics[height=0.25\textwidth]{line_characterization_vi_vout50_vrot100_logtau7}
\end{center}
\caption{\textbf{Valley Intensity. } We show for each \tauh the dependency that
  the viewing angle $\theta$ has on the line's the
  valley intensity. $\vrot=100$\kms is fixed for all panels.
		\label{fig:valley_intensity}}
\end{figure*}

\begin{figure}
\centering
    \includegraphics[width=0.48\textwidth]{doppler}
  \caption{\textbf{Spectra from receding/approaching sides of a toy
      model LAE.}  These results correspond to the RT simulation with
    $\vout=25$\kms, $\vrot=50$\kms, $\tauh=10^5$. 
    The spectra were computed for a viewing angle of
    $\theta=90^{\circ}$. This toy model illustrates to what extent spectra from
    opposite sides of a galaxy have an imprint of the rotational
    kinematics. 
    \label{fig:doppler}}
\end{figure}


\DIFdelbegin %DIFDELCMD < \begin{figure}
%DIFDELCMD < \centering
%DIFDELCMD < \includegraphics[width=0.45\textwidth]{muse_vs_simulations.pdf}
%DIFDELCMD < 

%DIFDELCMD < %%%
%DIFDELCMD < \caption{%
{%DIFAUXCMD
\textbf{\DIFdelFL{Comparison between simulations and observations.}}  
%DIFAUXCMD
\DIFdelFL{The observational results correspond to the MUSE-Wide Survey
\mbox{%DIFAUXCMD
\citep{2017A&A...606A..12H}}%DIFAUXCMD
. The skewness and standard deviation from
the simulations summarize the results presented in Figures
\ref{fig:standard_deviation} and \ref{fig:skewness}.
  }%DIFDELCMD < \label{fig:muse}%%%
}
%DIFAUXCMD
%DIFDELCMD < \end{figure}
%DIFDELCMD < 

%DIFDELCMD < %%%
\DIFdelend \section{Theoretical Models}
\label{sec:theory}


\DIFdelbegin \DIFdel{The Monte Carlo code we use (CLARA ) \mbox{%DIFAUXCMD
\citep{CLARA}}%DIFAUXCMD
}\DIFdelend \DIFaddbegin \DIFadd{We use CLARA \mbox{%DIFAUXCMD
\citep{CLARA}}%DIFAUXCMD
, a Monte Carlo code that }\DIFaddend follows the
propagation of individual photons through a neutral Hydrogen medium
characterized by its temperature, velocity field and global optical depth.
The code assumes \DIFdelbegin \DIFdel{a }\DIFdelend \DIFaddbegin \DIFadd{an }\DIFaddend homogeneous density throughout the simulated volume.
In the current implementation we neglect the influence of dust.
Our basic model is an spherical distribution of neutral hydrogen,
an approximation commonly used in the literature \DIFdelbegin \DIFdel{, as it }\DIFdelend \DIFaddbegin \DIFadd{that }\DIFaddend explains a
wide variety of observational features
\citep{Ahn03,Verhamme06,Dijkstra06}. 

\DIFdelbegin \DIFdel{The velocity field we use }\DIFdelend \DIFaddbegin \DIFadd{We use a velocity field that }\DIFaddend captures both outflows and rotation.
Outflows are described by a Hubble-like radial velocity profile with
the \DIFdelbegin \DIFdel{velocity magnitude }\DIFdelend \DIFaddbegin \DIFadd{speed }\DIFaddend increasing linearly with the radial coordinate; \DIFdelbegin \DIFdel{the outflows }\DIFdelend \DIFaddbegin \DIFadd{this }\DIFaddend model
is fully characterized by $v_{\rm out}$, the \DIFaddbegin \DIFadd{radial }\DIFaddend velocity at the
sphere's surface. 
Rotation follows a solid body rotation profile \DIFdelbegin \DIFdel{, which is }\DIFdelend fully characterized by
$v_{\rm rot}$ \DIFdelbegin \DIFdel{, }\DIFdelend the linear velocity at the sphere's surface.  

The total velocity field \DIFdelbegin \DIFdel{corresponds to }\DIFdelend \DIFaddbegin \DIFadd{is }\DIFaddend the superposition of rotation and
outflows.
The cartesian components take the following form:

\begin{equation}
	v_{x}=\frac{x}{R}\vout - \frac{y}{R}\vrot ,
	\label{eq:vx}
\end{equation}

\begin{equation}
	v_{y}=\frac{y}{R}\vout + \frac{x}{R}\vrot ,
	\label{eq:vy}
\end{equation}

\begin{equation}
	v_{z}=\frac{z}{R}\vout,
	\label{eq:vz}
\end{equation}
%
where $x$, $y$ and $z$ are the cartesian position coordinates with the
origin at the sphere's center, $R$ is the radius of the sphere and the
direction of the angular velocity vector \DIFdelbegin \DIFdel{corresponds to the }%DIFDELCMD < \hatk %%%
\DIFdel{unit vector}\DIFdelend \DIFaddbegin \DIFadd{is the $z$ axis}\DIFaddend .

For each \DIFdelbegin \DIFdel{model setup }\DIFdelend \DIFaddbegin \DIFadd{run }\DIFaddend we follow $10^5$ individual photons generated \DIFdelbegin \DIFdel{at
the
center of the sphere }\DIFdelend \DIFaddbegin \DIFadd{from the
origin }\DIFaddend at the \lya line's center as they propagate
through the volume and finally \DIFdelbegin \DIFdel{scape}\DIFdelend \DIFaddbegin \DIFadd{escape}\DIFaddend .
We store the \DIFdelbegin \DIFdel{final }\DIFdelend frequency and propagation direction for each photon
at its last scattering.

As input parameters we use $\tauh=\{10^5, 10^6, 10^7\}$,
$\vout=\{5,25,50\}\;\kms$ and $\vrot=\{0,50,100\}\;\kms$, for a total of
$27$ models with all the possible parameter combinations.
The \DIFdelbegin \DIFdel{range of values have some overlap with the expectations from a
galaxy with a total neutral hydrogen mass of $10^8$-$10^9$
$M_{\odot}$}\DIFdelend \DIFaddbegin \DIFadd{values for the outflow velocity are lower than values commonly used in the
literature to allow for an interplay between the two kinematic features.
This range of parameters also produce emission lines with standard deviation and
skewness in the same range to those of high redshift LAES in recent
observations \mbox{%DIFAUXCMD
\citep{2017A&A...606A..12H}}%DIFAUXCMD
}\DIFaddend .


\DIFaddbegin \DIFadd{We also define the viewing angle, $\theta$, as the angle
between the rotation axis and the line of sight of a potential
observer. 
}

\DIFaddend \cite{Garavito14} presented an analytical model that
accounts for the effects of pure rotation on the
\lya line morphology. 
The basic assumption of \DIFdelbegin \DIFdel{their analytical }\DIFdelend \DIFaddbegin \DIFadd{the }\DIFaddend model is that each differential surface
element on the sphere Doppler shifts the photons that it emits.
In this paper we introduce this ansatz by post-processing the results
of the outflows simulations without rotation.
The frequency of each photon is Doppler shifted as follows

\begin{equation}
x' = x + \frac{\vec{v}_{\rm rot}\cdot \hatk}{\vth}
\label{eq:shift_x}
\end{equation}
%
where $x'$ is the photon's new adimensional frequency, $x$ is the photon's
frequency after being processed \DIFaddbegin \DIFadd{only }\DIFaddend by the outflow, \vrot is the 
rotational velocity at the point of escape of the photon, \hatk is
the photon's direction of propagation and \vth is the thermal
velocity of the sphere. 
\DIFaddbegin \DIFadd{We fix }\vth \DIFadd{throughout the paper. This factor is a multiplying
constant that could be changed without running new Monte-Carlo
simulations, as the code works with the adimensional frequency $x$. 
}\DIFaddend 

This \DIFaddbegin \DIFadd{semi-analytic model }\DIFaddend allows us to produce new \lya spectra \DIFaddbegin \DIFadd{from
the outflow-only results }\DIFaddend and compare them with the full radiative
transfer solution including both outflows and rotation. 

\section{Results}
\label{sec:results}

\subsection{Qualitative Trends}
\label{sec:qualitative}

Figure \ref{fig:doppler_shift} summarizes the most important trends
\DIFaddbegin \DIFadd{from rhw RT simulations}\DIFaddend .
In the left side, the six panels correspond to $\tau=10^6$ and a viewing angle of
$\theta =90^{\circ}$, that is, perpendicular to the rotation axis of the
galaxy. 
In every panel the thin black line corresponds to the pure outflow
solution, i.e. without rotation. 
From top to bottom we see the effect of increasing the outflow
velocity, which is the expected increasing asymmetry towards the red
peak. 

The thick black line corresponds to the solution that includes both
outflows and rotation.
Comparing the left and right columns (lower versus higher rotational
velocity) we can see two immediate effects.
First, the line broadens and second, the intensity at the line's
center increases.

The thick gray line corresponds to the pure outflow solution
with the Doppler \DIFdelbegin \DIFdel{boost }\DIFdelend \DIFaddbegin \DIFadd{shift }\DIFaddend added to model rotation's influence.
At $\tauh=10^6$ the Doppler \DIFdelbegin \DIFdel{boost }\DIFdelend \DIFaddbegin \DIFadd{shift }\DIFaddend does a good job at capturing the broad
morphological features introduced by rotation: the angle dependence,
the broadening and the intensity increase at the line's center.

In the right side of Figure \ref{fig:doppler_shift} we show the same results as in the left one, but for a viewing angle of \DIFdelbegin \DIFdel{$\theta =
90^{\circ}$}\DIFdelend \DIFaddbegin \DIFadd{$\theta =
0^{\circ}$}\DIFaddend , that is parallel to the rotation axis. 
In this case we confirm the result presented by \cite{Garavito14},
namely that pure rotation introduces a strong dependence with 
viewing angle, a trend that we find also holds for rotation mixed with
outflows.   

The quality of the results from the Doppler \DIFdelbegin \DIFdel{boost }\DIFdelend \DIFaddbegin \DIFadd{shift }\DIFaddend improves for higher
\tauh values. 
In the Appendix we show the same plots as Figure
\ref{fig:doppler_shift}, there it is evident that for $\tauh=10^5$ the
results are not as good as they are for $\tauh=10^6$, and that for
$\tauh=10^7$ the Doppler \DIFdelbegin \DIFdel{boost
}\DIFdelend \DIFaddbegin \DIFadd{shift }\DIFaddend provides a remarkable good
approximation. 

\subsection{Quantitative \DIFdelbegin \DIFdel{trends}\DIFdelend \DIFaddbegin \DIFadd{Trends}\DIFaddend }
\label{sec:quantitative}

After finding the qualitative influence of the different parameters we
move onto a quantitative study.
To do this we summarize the line morphology by four different
\DIFdelbegin \DIFdel{characteristics}\DIFdelend \DIFaddbegin \DIFadd{scalars}\DIFaddend : standard deviation (STD), skewness (SKW), bimodality
(BI) and valley/peak ratio.
These quantities are defined by the following equations \citep{kokoska1999}:

\begin{equation}
\label{eq:std}
\STD = \sqrt{m_2},
\end{equation}

\begin{equation}
\label{eq:skw}
\SKW = \frac{m_3}{m_2^{3/2}},
\end{equation}

\begin{equation}
\label{eq:bi}
\BI = \mathrm{KURTOSIS} - \SKW^2 = \frac{m_4}{m_2^{2}} - \frac{m_3^2}{m_2^{3}},
\end{equation}
%
where each $m_i$ is the i-th moment about the mean. 
The STD has velocity units and quantifies the line's width.
The SKW is adimensional and quantifies the peaks' asymmetry. 
In the case of a bimodal distribution, $\SKW>0$ means that the blue
peak is taller and for $\SKW<0$ the red peak is taller. 
The BI is adimensional and quantifies whether the line has 1 or 2
peaks: it is  always $\geq 1$ \citep{Pearson1929} and the closer 
to 1, the more bimodal is the line (i.e. has 2 similar peaks). 
We found by visual inspection of our spectra that $\BI=2.5$ marks the
transition between two peaks (however imbalanced) and a dominant
single peak.


\subsubsection{Standard Deviation}
Figure \ref{fig:standard_deviation} summarizes the standard deviation
results for all our models.
Each panel shows the STD as a function of \vrot.
All panels were computed using a viewing angle of $\theta =
90^{\circ}$ (perpendicular to the rotation axis), which has the most
extreme influence from rotation.
 The black triangles
correspond to the full RT solution and the line to the DS
approximation.  
The optical depth increases from top to bottom and the outflow
velocity from left to right.
This quantitative plot confirms that the line width increases with
rotational velocity and optical depth.
These trends are expected; higher rotational velocities can be seen as
an addition of different Doppler shifts that smear out the line, while
a higher optical depth translates into a larger number of scatterings
that increase the probability of a photon to diffuse in frequency
resulting in a broader line.

The DS successfully reproduces all trends with the optical depth,
rotational velocity and outflow velocity.
However, the DS consistently underestimates the STD. 
The difference between the RT and DS increases with the outflow
velocity and the rotational velocity, and decreases with increasing
optical depth.
In the range of parameter space explore, this difference has as an
upper bound of $\sim 7\%$, $3\%$ and $\sim 2\%$ for
 $\tauh=10^5$, $10^6$ and $10^7$, respectively. 

\subsubsection{Skewness}

Figure \ref{fig:skewness} presents the skewness results for all the
models together with the DS comparison following the same layout as
Figure \ref{fig:standard_deviation}.
In all cases the skewness is negative showing that all the lines
are unbalanced towards the red side of the spectrum.
Skewness increases with rotational velocity and decreases with
optical depth; rotation tries to smooth the line diminishing the
asymmetries while a higher optical depth reinforces the line asymmetries.
The skewness does not have a monotonous trend with outflow velocity because
there is a transition between double and single peak line; for low
outflow velocities the skewness signals the balance between the two
existing peaks while for high outflow velocities it quantifies the
asymmetry of the already dominant read peak.

The DS reproduces the main trends, again with an underestimation that
decreases at higher optical depths and increases with larger values of
the rotational velocity and outflow velocity.
In this case the differences between RT and DS have an upper bound of
$85\%$, $35\%$ and $5\%$ 
for  $\tauh=10^5$, $10^6$ and $10^7$,
respectively.  


\subsubsection{Bimodality}

Figure \ref{fig:bimodality} shows the results for the bimodality using
the same layout as in the two previous Figures.
Following the reasoning about the skewness, we observe that
increasing the outflow velocity increases the value of bimodality,
that is, it transitions to a more pronounced single peak. 
The trend as a function of the rotational velocity and the optical
depth are not monotonous.
When the outflow velocity is low ($\vout<50\ \kms$), an increasing
rotational velocity smears the two asymmetrical peaks pushing the line
morphology towards a single peaks, making the bimodality statistics
increase. 
On other situations ($\vout=50\ \kms$ and $\tauh\geq 10^6$) higher
rotational velocities the bimodality statistics decreases, which means
that it manages to slightly enlarge the already dominant red peak.

The DS reproduces the main trends while underestimating the bimodality
statistics. 
As expected from the previous results the difference between RT and DS
decreases at higher optical depths and increases with increasing
values of the rotational and outflow velocities.
In this case the differences have an upper bound of $4\%$, $2\%$ and
$1\%$ for  $\tauh=10^5$, $10^6$ and $10^7$, respectively.  

\subsubsection{Intensity at line's center}

In Figure \ref{fig:valley_intensity} we quantify how the intensity at
the line's center (i.e. the valley) changes with the viewing angle,
the outflow velocity and the optical depth.
These results correspond to a fixed rotational velocity of
$\vrot=100\ \kms$.
The triangles correspond to the RT simulations and the line represents
the DS results. 
The valley intensity is expressed as a fraction of the maximum peak
intensity in the line, as such the valley/peak ratio is always $<1$. 
In every panel we see that the valley/peak ratio decreases as the
observer moves from a line  of sight perpendicular to the rotation
axis onto a parallel line of sight. 
This is a clear demonstration of the viewing angle dependency
introduced by rotation.

The valley/peak ratio at $\cos{\theta}=1$ matches results
without rotation, this shows that for increasing rotational velocity
the valley/peak ratio increases.
In turn, for increasing optical depth or outflowing velocity this
ratio decreases.
Once again, the DS results correctly follow the trends for the full RT
simulations.  
This time the differences have an upper bound of $55\%$, $2\%$ and
$1\%$ for  $\tauh=10^5$, $10^6$ and $10^7$, respectively.  


\section{Discussion}
\label{sec:discussion}

In this section we discuss how the results we have presented can be
connected to the interpretation of observational data.


\subsection{\DIFdelbegin %DIFDELCMD < \lya %%%
\DIFdel{continuum leakage or rotation?}\DIFdelend \DIFaddbegin \DIFadd{The semi-analytic model as gaussian smoothing}\DIFaddend }

\DIFdelbegin \DIFdel{One application of this new model to the current interpretation of }%DIFDELCMD < \lya
%DIFDELCMD < %%%
\DIFdel{spectra
, is that the central ($V=0$) emission of the spectra seen in the
}%DIFDELCMD < \lya %%%
\DIFdel{line, 
is a }\DIFdelend \DIFaddbegin \DIFadd{The effects of the semi-analytic model on the pure outflowing spectra
are similar to the expected results from a gaussian smoothing, 
such smoothing can also be a natural }\DIFaddend consequence of the \DIFdelbegin \DIFdel{viewing angle of the galaxy and can be
controlled by it.
Several authors have suggested that this central emission is caused by
radiation that escapes the galaxy without scattering, mostly due to a clumpy interstellar medium \mbox{%DIFAUXCMD
\citep{Hansen06, 2016ApJ...833L..26G}}%DIFAUXCMD
.
Rotation is an alternative that solves this issue}\DIFdelend \DIFaddbegin \DIFadd{intruments
used to measure the LAE spectra.
As a consequence, natural experimental artifacts could be mistaken as
an indication for rotation}\DIFaddend .

\DIFdelbegin \DIFdel{We note that the relative intensity of the two peaks in the outflowing
spectra is not modified by rotation , that asymmetry is solely
controlled by the outflow velocity.
This means that results that have already derive a
typical outflowing
velocity as  to match the observational constraints do not have to be
completely revised. 
Only an additional exploration of rotational velocities and
viewing
angles needs to be explored.
}\DIFdelend \DIFaddbegin \DIFadd{To understand to what extent the effects of rotation can be seen as a
simple smoothing, we model the effects of gaussian smoothing with a
similar ansatz as  the semi-analytic model by changing the frequency
of each photon by a Gaussian random variable centered on zero and
standard deviation $\sigma_x$: }\begin{equation}
\DIFadd{x^{\prime} = x + \mathcal{N}(0,\sigma_x^2).
}\end{equation}
\DIFaddend 

\DIFdelbegin \DIFdel{A recent observational example is provided by
\mbox{%DIFAUXCMD
\cite{2017A&A...608L...4R}}%DIFAUXCMD
.
They report optical spectroscopy of a bright lensed galaxy at a redshift of $z=2.4$. 
The object presents a triple peaked }%DIFDELCMD < \lya %%%
\DIFdel{profile that can be thought
as the superposition of two components, a double asymmetric peak and a
narrower symmetric peak around the line's center. 
One possible option for the narrow component is that it escaped the
galaxy without substantial interaction with neutral Hydrogen. 
We speculate that the central feature could also be due to a purely 
rotating component that can produce such a feature as shown by
\mbox{%DIFAUXCMD
\cite{tololo}}%DIFAUXCMD
. 
The triple peak could be then produced by two main
decopled components: one rotating the other outflowing.
}\DIFdelend \DIFaddbegin \DIFadd{We find that the results of the semi-analytic model on the total
spectra at a given angle $\theta$ can be approximated using
$\sigma_x = (1/2)\times(\vrot/\vth)\times \sin\theta$. 
This approximation reproduces within a few porcent all quantities
measured in Section \ref{sec:quantitative} for the semi-analytic
model, it only works well for $\tau=10^6$ and $\tau=10^7$. 
}\DIFaddend 

\DIFdelbegin \DIFdel{Another interesting example are the }%DIFDELCMD < \lya %%%
\DIFdel{lines in Green Pea galaxies \mbox{%DIFAUXCMD
\citep{2016ApJ...820..130Y}}%DIFAUXCMD
.
\mbox{%DIFAUXCMD
\cite{2017ApJ...844..171Y} }%DIFAUXCMD
showed in a
study of 43 Green Pea galaxies
that 2/3 were strong }%DIFDELCMD < \lya %%%
\DIFdel{emitters.  
There the }%DIFDELCMD < \lya %%%
\DIFdel{escape fraction is effectively estimated from the
ratio
of H$\alpha$ to }%DIFDELCMD < \lya%%%
\DIFdel{.
Galaxies with large escape fractions also substantial }%DIFDELCMD < \lya %%%
\DIFdel{emission at
the line's center.
We speculate that this feature could help to constrain rotation in
Green Pea galaxies. 
}\DIFdelend \DIFaddbegin \DIFadd{This result can be used as an order of magnitude estimate of the mimum
espectral resolution required to observe the effects of a rotational
velocity of $\vrot$. 
The Space Telescope Imaging Spectrograph (STIS) on the Hubble Space
Telescope (HST) can provide resolutions around 20 }\kms
\DIFadd{\mbox{%DIFAUXCMD
\citep{2015PASA...32...27H}}%DIFAUXCMD
, which could allow for the detection of
rotation features in nearby galaxies with at least $40$ }\kms \DIFadd{and a
rotation axis perpendicular to the line of sight.  
Other instruments such as the Cosmic Origins Spectrograph (COS) have
varying resolutions between $20$}\kms \DIFadd{and $200$}\kms \DIFadd{according to the
angular size of the source \mbox{%DIFAUXCMD
\citep{2018A&A...616A..60O}}%DIFAUXCMD
.
}\DIFaddend 

\subsection{\lya Kinematic Maps}
\DIFaddbegin \label{subsec:kinematic}
\DIFaddend 

Current observational facilities have the capability of spatially
resolving the extent of a LAE.
For instance \cite{Prescott14} presented observational results of a
Doppler shift when taking spectra at two opposite sides of a large
($\approx 80$kpc) LAE.
In more recent work \cite{2018MNRAS.473.3907A} mapped \lya emission
around a quasar.
\DIFaddbegin \DIFadd{In this case one could measure the displacement between the
peaks in the two different spectra. 
The interest of this test is that the location of the peaks only
depends on the kinematic properties of the sources and does not
change by convolution with a gaussian window function due to limited
instrumental resolution, as discussed in the previous section.
}\DIFaddend 

In Figure \ref{fig:doppler} we present a toy model (\vrot$=50\ \kms$,
\vout$=25\ \kms$ and \tauh$=10^5$) for the spectrum of
a LAE taken from to different sides of the galaxy. 
As the LAE is rotating, one side is being redshifted while the other
is blueshifted. 
We see that the full spectrum is a weighted line, in solid black,
that is found between these two.
We notice that the distance between the maxima of the blue and red
spectra is not twice the rotational velocity as it could be naively
expected.

\DIFdelbegin \DIFdel{Although it is a good approximation to think the rotating spectra by a
sum of Doppler shifts, the peak of the spectra is also weighted by the
amount of mass with a given velocity. 
}\DIFdelend In this toy model the distance between the peaks of the
receding/approaching spectra is close to \DIFdelbegin \DIFdel{$\sim 20\ \kms$}\DIFdelend \DIFaddbegin \DIFadd{$\sim 25\ \kms$}\DIFaddend , which is
\DIFdelbegin \DIFdel{less than half }\DIFdelend \DIFaddbegin \DIFadd{a fourth }\DIFaddend of the naively expected value of $2\vrot=100\ \kms$,
due to the fact that only a small fraction of the photons are emitted
at the extreme of the galaxy having the maximum rotational velocity of
$50\ \kms$. 
\DIFaddbegin \DIFadd{Although it is a good approximation to think the rotating spectra by a
sum of Doppler shifts, the peak of the spectra is also weighted by the
amount of mass with a given line-of-sight velocity. 
}\DIFaddend 

\DIFdelbegin \DIFdel{New spectrographs like MUSE should help }\DIFdelend \DIFaddbegin \DIFadd{Spectrographs like the Multi Unit Spectroscopic Explorer (MUSE) could
}\DIFaddend obtain kinematic information from \DIFaddbegin \DIFadd{large samples of }\DIFaddend LAEs to build
velocity maps in Ly$\alpha$.  
This could be a natural extension of the work reported by
\cite{Herenz2016} on the velocity maps of several LARS (Lyman Alpha
Reference Sample) galaxies.  
The interpretation of such data \DIFdelbegin \DIFdel{should }\DIFdelend \DIFaddbegin \DIFadd{could }\DIFaddend take into account the insights
and trends we have presented in this paper.


\DIFdelbegin \subsection{\DIFdel{Comparison against the MUSE-Wide Survey}}
%DIFAUXCMD
\addtocounter{subsection}{-1}%DIFAUXCMD
%DIFDELCMD < 

%DIFDELCMD < %%%
\DIFdel{In order to give the reader a feeling of the ranges of observational
data that could be explained by the models presented here, 
we perform a comparison against observational data from the MUSE-Wide
Survey (MWS) \mbox{%DIFAUXCMD
\citep{2017A&A...606A..12H}}%DIFAUXCMD
.   
}%DIFDELCMD < 

%DIFDELCMD < %%%
\DIFdel{MWS provides public access to 831 spectra out of which 237 contain a
detected }%DIFDELCMD < \lya %%%
\DIFdel{line as dominant. 
The public dataset reports as the redshift of the galaxy the position
of the dominant peak.
We use this redshift to convert the spectra to a rest-frame and compute
the standard deviation and skewness statistics. 
Given the uncertainty on the galaxy redshift we do not compute the
valley/peak ratio. 
We do not try to perform a full fit to all the spectra
\mbox{%DIFAUXCMD
\citep[e.g.][]{2017A&A...608A.139G} }%DIFAUXCMD
and wish only to
know whether the region of parameter space explored in our RT
simulations shows any crossing with the results from the MWS or
shows any sign of atypical features.
 }%DIFDELCMD < 

%DIFDELCMD < %%%
\DIFdel{Figure \ref{fig:muse} shows the results of this experiment. 
The small white circles correspond to the 217 }%DIFDELCMD < \lya %%%
\DIFdel{emitters in the MWS
observations; the large black circles to the results of the
simulations presented in this paper. 
This shows that the systems modeled in this paper actually present an
overlap with statistics derived from observations. 
The region with standard deviation $>200$}%DIFDELCMD < \kms %%%
\DIFdel{could also be explained
by systems with larger optical depths than we have used here. 
However, the systems with a positive skewness, that is a line
unbalanced towards the blue could be the sign of inflowing
kinematics. 
}%DIFDELCMD < 

%DIFDELCMD < %%%
\DIFdelend \section{Conclusions}
\label{sec:conclusions}

In this paper we explore, for the first time in the literature,
the results of a model for the emergent \lya line from rotating
outflows. 
\DIFdelbegin \DIFdel{The main results for the model are computed from }\DIFdelend \DIFaddbegin \DIFadd{We the }\DIFaddend a \DIFaddbegin \DIFadd{semi-analytic model first presented by \mbox{%DIFAUXCMD
\cite{Garavito14}
}%DIFAUXCMD
to capture the main effects of rotation and confront it against
results from }\DIFaddend Monte-Carlo radiative transfer \DIFdelbegin \DIFdel{simulation and confronted with a simple
}\DIFdelend \DIFaddbegin \DIFadd{simulations. 
The }\DIFaddend semi-analytic \DIFdelbegin \DIFdel{ansatz that adds the effects of rotation onto results of pure outflow kinematics}\DIFdelend \DIFaddbegin \DIFadd{model only takes into account the Doppler shift computed
as product of quantities at the surface of last scattering, namely
$\vec{v}_{\rm rot}\cdot \hatk$, where $\vec{v}_{\rm rot}$ is the
velocity due to rotation and }\hatk \DIFadd{is the direction of the photon's
propagation}\DIFaddend . 

To address the first question we posed in the introduction (\emph{\DIFdelbegin \DIFdel{what }\DIFdelend \DIFaddbegin \DIFadd{What
  }\DIFaddend is \DIFdelbegin \DIFdel{then
}\DIFdelend the expected imprint of rotation on a resonant emission line such
  as the \lya line?}) we \DIFdelbegin \DIFdel{can say that the
main }\DIFdelend \DIFaddbegin \DIFadd{find from the full RT simulations that the
}\DIFaddend effects of rotation on the \lya line morphology are:  

\begin{itemize}
  \item Inducing a dependency on the viewing angle.
  \item Broadening the line.
  \item Increasing the intensity at the line's center.
\end{itemize}
\DIFdelbegin %DIFDELCMD < 

%DIFDELCMD < %%%
\DIFdel{We find that }\DIFdelend %DIF > 
\DIFaddbegin \DIFadd{All }\DIFaddend these effects can be \DIFdelbegin \DIFdel{quantitatively explained by a
Doppler shift computed as product of quantities at the surface of last
scattering, namely  $\vec{v}_{\rm rot}\cdot \hatk$, where
$\vec{v}_{\rm rot}$ is the velocity due to rotation and }%DIFDELCMD < \hatk %%%
\DIFdel{is
the direction of the photon's propagation. 
These conclusions, specially the confirmation of the Doppler Boost as
a good approximation to model solid body rotation, strengthen the
}\DIFdelend \DIFaddbegin \DIFadd{qualitatively explained by the proposed
}\DIFaddend semi-analytic \DIFaddbegin \DIFadd{model. 
Quantitatively speaking the semi-analytic }\DIFaddend model \DIFdelbegin \DIFdel{reported by \mbox{%DIFAUXCMD
\cite{Garavito14} }%DIFAUXCMD
where they modeled
the effect of pure rotation on }%DIFDELCMD < \lya %%%
\DIFdel{spectra}\DIFdelend \DIFaddbegin \DIFadd{provides a
satisfactory answer for a neutral Hydrogen optical depth equal or
larger than $10^6$}\DIFaddend .  



Addressing the second question we posed in the introduction (\emph{\DIFdelbegin \DIFdel{to }\DIFdelend \DIFaddbegin \DIFadd{To }\DIFaddend what
extent is it possible to constrain rotational kinematics from the \lya
emission line?})  we \DIFdelbegin \DIFdel{can say that as an application to observational data our results suggest that
rotation should also be considered as an explanation of the non-vanishing }\DIFdelend \DIFaddbegin \DIFadd{show that the most straightforward approach
is measuring }\DIFaddend \lya \DIFdelbegin \DIFdel{intensity at the line's center, which could change
the interpretation of recent observational results
\mbox{%DIFAUXCMD
\citep[e.g.][]{2017ApJ...844..171Y,2017A&A...608L...4R} }%DIFAUXCMD
that claim
instead that this flux constrains the escape fraction of }%DIFDELCMD < \lya
%DIFDELCMD < %%%
\DIFdel{radiation. 
}%DIFDELCMD < 

%DIFDELCMD < %%%
\DIFdel{A more straightforward approach
takes into account how the
}%DIFDELCMD < \lya %%%
\DIFdelend spectra from two different \DIFdelbegin \DIFdel{sides }\DIFdelend \DIFaddbegin \DIFadd{regions }\DIFaddend of a
galaxy \DIFdelbegin \DIFdel{can }\DIFdelend \DIFaddbegin \DIFadd{to }\DIFaddend detect approaching/\DIFdelbegin \DIFdel{receding gas motions
}\DIFdelend \DIFaddbegin \DIFadd{receeding gas motions
\mbox{%DIFAUXCMD
\citep{Prescott14,2018MNRAS.473.3907A}}%DIFAUXCMD
}\DIFaddend . 
In that case, \DIFdelbegin \DIFdel{one has to be careful and take into account that the peaks of these two spectra is smaller than the }\DIFdelend \DIFaddbegin \DIFadd{we also show that the distances between these two peaks 
is four times shorter than the naively }\DIFaddend expected value of
$2\times\vrot$\DIFdelbegin \DIFdel{due to the different }\DIFdelend \DIFaddbegin \DIFadd{.  
This difference is produced by the different geometric }\DIFaddend weights at the
\DIFdelbegin \DIFdel{surface of the
emitting regions.    
}%DIFDELCMD < 

%DIFDELCMD < %%%
\DIFdel{We also use data from MUSE to demonstrate to what extent the outputs of the RT results presented here can be expected to match
hundreds of observed }%DIFDELCMD < \lya %%%
\DIFdel{spectra, but a full constraint of rotation
properties certainly require a full Markov Chain Monte Carlo
exploration of parameter space}\DIFdelend \DIFaddbegin \DIFadd{emitting surface.    
Here the semi-analytic model also provides the correct quantitative
insight}\DIFaddend . 

\DIFdelbegin \DIFdel{Our work confirms that rotation produces features in the }%DIFDELCMD < \lya %%%
\DIFdel{line
strong enough that can be expected to be detected in observations and
that should be taken into account into }%DIFDELCMD < \lya %%%
\DIFdel{modeling; the Doppler Boost }\DIFdelend \DIFaddbegin \DIFadd{To summarize, our works shows that the Doppler Shift }\DIFaddend offers an
easy-to-implement approximation to explore such influence into already
existing \DIFdelbegin \DIFdel{RT simulations and provide stronger
footing to the interpretation of }\DIFdelend \DIFaddbegin \DIFadd{radiative transfer simulation simulations that include
outflows, providing a versatile tool to interpret current and future }\DIFaddend \lya
kinematics maps  \citep[e.g][]{2018MNRAS.473.3907A}.   

\bibliographystyle{mnras}
\bibliography{references}



\DIFdelbegin %DIFDELCMD < \newpage
%DIFDELCMD < %%%
\DIFdelend \appendix

\section{Additional figures}
\label{sec:appendix}

\DIFdelbegin %DIFDELCMD < \begin{figure}
%DIFDELCMD <   %%%
\DIFdelendFL \DIFaddbeginFL \begin{figure*}
  \DIFaddendFL \begin{center}
    \includegraphics[width=0.49\textwidth]{doppler_shift_logtau5_theta90}
    \DIFdelbeginFL %DIFDELCMD < \end{center}
%DIFDELCMD <   %%%
%DIFDELCMD < \caption{%
{%DIFAUXCMD
\textbf{\DIFdelFL{Qualitative trends of changing outflow and
      rotational velocity.}}
    %DIFAUXCMD
\DIFdelFL{Same layout as Figure \ref{fig:doppler_shift},
    this time  $\tauh=10^5$ and $\theta=90^\circ$.}}
%DIFAUXCMD
%DIFDELCMD < \end{figure}
%DIFDELCMD < 

%DIFDELCMD < \begin{figure}
%DIFDELCMD <   \begin{center}
%DIFDELCMD <     %%%
\DIFdelendFL \includegraphics[width=0.49\textwidth]{doppler_shift_logtau7_theta90}
  \end{center}
  \caption{\textbf{Qualitative trends of changing outflow and
      rotational velocity.}
    Same layout as Figure \ref{fig:doppler_shift}\DIFdelbeginFL \DIFdelFL{,
    }\DIFdelendFL \DIFaddbeginFL \DIFaddFL{. On the left:
    $\tauh=10^5$ and $\theta=90^\circ$; on the right:     }\DIFaddendFL this time  $\tauh=10^7$ and $\theta=90^\circ$.}
\DIFdelbeginFL %DIFDELCMD < \end{figure}
%DIFDELCMD < %%%
\DIFdelend \DIFaddbegin \end{figure*}
\DIFaddend 

\end{document}
