\documentclass[a4paper,fleqn,usenatbib]{mnras}
%=========================================================================
\usepackage{amsmath}
\usepackage{amssymb}
\usepackage{graphicx}
\usepackage{grffile}
\usepackage[dvips]{epsfig}
\usepackage{epsfig}
\usepackage{color}
\usepackage{caption}
\usepackage{hyperref}
\usepackage{bm}
%Non reposionated tables
\newcommand{\HI}{{\text{H\MakeUppercase{\romannumeral 1}}} }
\newcommand{\lya}{\ifmmode{{\rm Ly}\alpha}\else Ly$\alpha$\ \fi}
\newcommand{\kms}{\ifmmode\mathrm{km\ s}^{-1}\else km s$^{-1}$\fi}
\newcommand{\vrot}{\ifmmode v_{\mathrm{rot}}\else $v_{\mathrm{rot}}$~\fi}
\newcommand{\vout}{\ifmmode v_{\mathrm{out}}\else $v_{\mathrm{out}}$~\fi}
\newcommand{\tauh}{\ifmmode \tau_{\mathrm{H}}\else $\tau_{\mathrm{H}}$~\fi}
\newcommand{\vth}{\ifmmode v_{\mathrm{th}}\else $v_{\mathrm{th}}$~\fi}

\begin{document}

%=========================================================================
%		FRONT MATTER
%=========================================================================
\title[Outflows and rotation in LAEs]{Lyman-$\alpha$ photons through rotating outflows}
\author[M.C. Remolina-Gutierrez \& J.E. Forero-Romero]{
  Maria Camila Remolina-Guti\'errez$^{1}$
  \thanks{mc.remolina197@uniandes.edu.co} \&
  Jaime E. Forero-Romero $^{1}$
  \thanks{je.forero@uniandes.edu.co}\\
  %%
  $^{1}$ Departamento de F\'isica, Universidad de los Andes, Cra. 1
  No. 18A-10 Edificio Ip, CP 111711, Bogot\'a, Colombia \\
}

\maketitle

\begin{abstract}
Outflows and rotation are two ubiquitous kinematic features in the gas
kinematics of galaxies.
Here we perform Monte Carlo radiative transfer simulations of outflowing
gas with additional solid body rotation to understand how these kinematic
features impact the morphology of the Lyman-$\alpha$ emission line.
We explore a range of Hydrogen optical depth of
$10^5\leq\tauh\leq10^7$, rotational velocity $0\leq \vrot/\kms \leq 100$ and
outflow velocity $0\leq \vout/\kms\leq 50$.  
We find three important consequences of rotation.
First, it introduces a dependency with viewing angle; second it
produces a line broadening and third it increases the flux at the
line's center.
We also develp a semi-analytic model that modifies the spectra of
pure outflow simulations that manages to reproduce the line broadening
and flux change at the line's center within a $7\%$ and $30\%$
precision, respectively.
\end{abstract}

\begin{keywords}
galaxies: dwarf --- radiative transfer --- Methods: numerical
\end{keywords}


%=========================================================================
%		PAPER CONTENT
%=========================================================================

%*************************************************************************

\section{Introduction}
\label{sec:intro}

The interpretation of the Lyman-$\alpha$ emission in galaxies is
key to understand their evolutionary processes, specially in
star-forming, low-dust galaxies \citep{PartridgePeebles}.
Recent improvements in instrumentation have revolutionized the kind of
studies that can be performed on Lyman-$\alpha$ emitting galaxies
(LAEs.)
For instance, it is now possible to infer detailed kinematic maps for
nearby galaxies.
The study of these maps would allow us to build data-driven models to
interpret the \lya spectra of unresolved galaxies, helping us to
constrain the physical conditions of the interstellar medium (ISM)
processing the \lya radiation.


For instances, dwarf galaxies show signs of coherent rotation
\citep{2009A&A...493..871S}.
Some of them are expected to have high neutral gas contents and thus
show \lya\ emission with rotation imprints
\citep{2005A&A...433L...1B,2008ApJ...672..888T,2013MNRAS.434.2491G}
Some Compact Dwarf Galaxies actually show gas kinematics that range
from pure rotation to high velocity dispersion without a clear
rotation pattern.
\citep{2015A&A...577A..21C,2017A&A...600A.125C}

The case for outflows as traced in the \lya\ line is much better
stablished.
In many cases the \lya line profiles has a single peak
redwards from the line's center, in other cases there is a double peak
but the peak on the red side is stronger
\citep[e.g.][]{2010ApJ...717..289S,Erb14,Trainor16}.   
These features been readily explained as the consequence of multiple
\lya photon scatterings through an homogeneous outflowing shell of
neutral Hydrogen
\citep{2006A&A...460..397V,Orsi12,2012ApJ...751...29Y,2015ApJ...812..123G}.  


Here we present for the first time a study of the joint effects of
galaxy outflows and rotation, two of the three major kinematic
components expected in LAEs.
We study a simplified geometrical configuration corresponding to a
spherical gas cloud with symmetrical radial outflows and a rotation
profile corresponding to a solid body.
We base our modeling on a Monte-Carlo radiative transfer code called
CLARA (Code for Lyman Alpha Radiation Analysis) presented for the
first time by \cite{CLARA}.

Besides modeling the impact of joint rotation and outflows, we also
check to what extent the analytical model presented by
\cite{Garavito14} to explain the effects of rotation can also be
applied in our case.
We show how solid body rotation effects can be modelled by
postprocessing the results of outflow only simulation.

in the results of other radiative transfer model in the case of large
optical depths.
In this case it is a good approximation to Doppler
boost the results of the model without rotation.

The structure of the paper is the following.
We introduce first our theoretical tools and assumptions
in Section \ref{sec:theory}. We continue in Section  \ref{sec:results}
with the results from the Monte-Carlo simulation, the comparison
against the semi-analytical approximation which we use to make a
thorough exploration of the effect of rotation.
In Section \ref{sec:discussion} we discuss our results and their
possible implications for observational analysis to finally present
our conclusions in Section \ref{sec:conclusions}.


\section{Theoretical Models}
\label{sec:theory}


The Monte Carlo code we use (CLARA)  follows the propagation of
individual photons through a neutral Hydrogen medium characterized by
its temperature, velocity field and global optical depth.
The code assumes an homogeneous density throughout the simulated
volume.
In the current implementation we neglect the influence of dust.
Our basic model is an spherical distribution of neutral hydrogen,
an approximation commonly used in the literature, as it explains a
wide variety of observational features
\citep{Ahn03,Verhamme06,Dijkstra06}. 


The velocity field we use captures both outflows and rotation.
Outflows are described by a Hubble-like radial velocity profile with
the velocity magnitude increasing linearly with the radial
coordinate; the outflows model is fully characterized by $v_{\rm out}$, the
velocity at the sphere's surface.
Rotation follows a solid body rotation profile, which is fully
characterized by $V_{\rm rot}$, the linear velocity at the sphere's surface.

The total velocity field corresponds to the superposition of rotation and
outflows.
The cartesian components take the following form:

\begin{equation}
	v_{x}=\frac{x}{R}V_{\rm out}-\frac{y}{R}V_{\rm rot} ,
	\label{eq:vx}
\end{equation}

\begin{equation}
	v_{y}=\frac{y}{R}V_{\rm out}+\frac{x}{R}V_{\rm rot} ,
	\label{eq:vy}
\end{equation}

\begin{equation}
	v_{z}=\frac{z}{R}V_{\rm out},
	\label{eq:vz}
\end{equation}
%
where $x$, $y$ and $z$ are the cartesian position coordinates with the
origin at the sphere's center, $R$ is the radius of the sphere and the
direction of the angular velocity vector corresponds to the $\hat{k}$
unit vector.

For each model setup we follow $10^5$ individual photons generated at
the center of the sphere at the \lya\ line's center as they propagate
through the volume and finally scape.
We store the final frequency and propagation direction for each photon
at its last scattering.

In Table \ref{tab:values} we list the combination of \tauh,
\vrot and \vrot values used in this paper.
The range of values have some overlap with the expectations from a
galaxy with a total neutral hydrogen mass of $10^8$-$10^9$
$M_{\odot}$. We run a total of $27$ different models. 


\cite{Garavito14} presented an analytical model that
accounts for the effects of pure rotation on the
\lya\ line morphology. 
The basic assumption of their analytical model is that each
differential surface element on the sphere Doppler shifts the photons
that it emits.
In this paper we introduce this ansatz by post-processing the results
of the outflows simulations without rotation.
The frequency of each photon is Doppler shifted as follows

\begin{equation}
x' = x + \frac{\vec{V}_{\rm rot}\cdot\hat{k}}{v_{\rm th}}
\label{eq:shift_x}
\end{equation}
%
where $x'$ is the photon's new frequency, $x$ is the photon's
frequency after being procesed by the outflow, $v_{\rm rot}$ is the 
rotational velocity at the point of escape of the photon, $\hat{k}$ is
the photon's direction of propagation and $v_{\rm th}$ is the thermal
velocity of the sphere.

This allows us to produce new \lya spectra and compare them with the
full radiative transfer solution including both outflows and
rotation.

\section{Results}

\begin{figure}
  \begin{center}
    \includegraphics[width=0.49\textwidth]{./figures/results/doppler_shift_logtau6_theta90}
  \end{center}
  \caption{\textbf{Qualitative trends of changing outflow and
      rotational velocity viewed perpendicular to the rotation axis}. 
    Here we fix $\tauh=10^6$ and $\theta=90^\circ$.
    We vary \vrot increasing from left to right and \vout increasing
    from top to bottom. 
    The thin black line corresponds to the \lya line obtained with
    CLARA without any rotation and the indicated outflow velocity.
    The thick black line corresponds to the results including both
    outflows and rotation.
    The thick gray line shows the results of modifyin the pure outflow
    solution by the Doppler shift presented in Equation \ref{eq:shift_x}
    (in thin line), if there is a radiative transfer of rotation and outflows
    (thick and clear line), and if there is a radiative transfer of
    only outflows, but also a Doppler shift from the rotational
    velocity (thick and dark line).  
    \label{fig:doppler_shift}}
\end{figure}

\begin{figure}
  \begin{center}
    \includegraphics[width=0.49\textwidth]{./figures/results/doppler_shift_logtau6_theta0}
  \end{center}
  \caption{\textbf{Qualitative trends of changing outflow and
      rotational velocity viewed parallel to the rotation axis}.     
    Same layout as Figure \ref{fig:doppler_shift},
    this time the viewing angle corresponds to $\theta=0^\circ$.
    The effects of rotation on
    the  \lya\ line morphology are not visible when viewed parallel to
    the rotation axis.
    \label{fig:doppler_shift_theta_0}}
\end{figure}




\begin{figure*}
\begin{center}
\includegraphics[height=0.27\textwidth]{./figures/results/line_characterization_std_vout5_logtau5.pdf}
\includegraphics[height=0.27\textwidth]{./figures/results/line_characterization_std_vout25_logtau5.pdf}
\includegraphics[height=0.27\textwidth]{./figures/results/line_characterization_std_vout50_logtau5.pdf}
\includegraphics[height=0.27\textwidth]{./figures/results/line_characterization_std_vout5_logtau6.pdf}
\includegraphics[height=0.27\textwidth]{./figures/results/line_characterization_std_vout25_logtau6.pdf}
\includegraphics[height=0.27\textwidth]{./figures/results/line_characterization_std_vout50_logtau6.pdf}
\includegraphics[height=0.27\textwidth]{./figures/results/line_characterization_std_vout5_logtau7.pdf}
\includegraphics[height=0.27\textwidth]{./figures/results/line_characterization_std_vout25_logtau7.pdf}
\includegraphics[height=0.27\textwidth]{./figures/results/line_characterization_std_vout50_logtau7.pdf}
\end{center}
\caption{\textbf{Standard Deviation trends.} Results for all the
  Radiative Transfer simulations (in triangles) compares against the
  Doppler Shift model (lines).
  \label{fig:standard_deviation}}
\end{figure*}

\begin{figure*}
\begin{center}
\includegraphics[height=0.27\textwidth]{./figures/results/line_characterization_skw_vout5_logtau5}
\includegraphics[height=0.27\textwidth]{./figures/results/line_characterization_skw_vout25_logtau5}
\includegraphics[height=0.27\textwidth]{./figures/results/line_characterization_skw_vout50_logtau5}
\includegraphics[height=0.27\textwidth]{./figures/results/line_characterization_skw_vout5_logtau6}
\includegraphics[height=0.27\textwidth]{./figures/results/line_characterization_skw_vout25_logtau6}
\includegraphics[height=0.27\textwidth]{./figures/results/line_characterization_skw_vout50_logtau6}
\includegraphics[height=0.27\textwidth]{./figures/results/line_characterization_skw_vout5_logtau7}
\includegraphics[height=0.27\textwidth]{./figures/results/line_characterization_skw_vout25_logtau7}
\includegraphics[height=0.27\textwidth]{./figures/results/line_characterization_skw_vout50_logtau7}
\end{center}
\caption{\textbf{Skewness for RT and DS:} The triangles represent the Radiative
    Transfer and the lines represent the Doppler Shift.
    \label{fig:skewness}}
\end{figure*}

\begin{figure*}
\begin{center}
\includegraphics[height=0.28\textwidth]{./figures/results/line_characterization_bi_vout5_logtau5}
\includegraphics[height=0.28\textwidth]{./figures/results/line_characterization_bi_vout25_logtau5}
\includegraphics[height=0.28\textwidth]{./figures/results/line_characterization_bi_vout50_logtau5}
\includegraphics[height=0.28\textwidth]{./figures/results/line_characterization_bi_vout5_logtau6}
\includegraphics[height=0.28\textwidth]{./figures/results/line_characterization_bi_vout25_logtau6}
\includegraphics[height=0.28\textwidth]{./figures/results/line_characterization_bi_vout50_logtau6}
\includegraphics[height=0.28\textwidth]{./figures/results/line_characterization_bi_vout5_logtau7}
\includegraphics[height=0.28\textwidth]{./figures/results/line_characterization_bi_vout25_logtau7}
\includegraphics[height=0.28\textwidth]{./figures/results/line_characterization_bi_vout50_logtau7}
\end{center}
\caption{\textbf{Bimodality for RT and DS:} The triangles represent the Radiative
	Transfer and the lines represent the Doppler Shift.
	\label{fig:bimodality}}
\end{figure*}


\begin{figure*}
\begin{center}
\includegraphics[height=0.28\textwidth]{./figures/results/line_characterization_vi_vout5_vrot100_logtau5}
\includegraphics[height=0.28\textwidth]{./figures/results/line_characterization_vi_vout25_vrot100_logtau5}
\includegraphics[height=0.28\textwidth]{./figures/results/line_characterization_vi_vout50_vrot100_logtau5}
\includegraphics[height=0.28\textwidth]{./figures/results/line_characterization_vi_vout5_vrot100_logtau6}
\includegraphics[height=0.28\textwidth]{./figures/results/line_characterization_vi_vout25_vrot100_logtau6}
\includegraphics[height=0.28\textwidth]{./figures/results/line_characterization_vi_vout50_vrot100_logtau6}
\includegraphics[height=0.28\textwidth]{./figures/results/line_characterization_vi_vout5_vrot100_logtau7}
\includegraphics[height=0.28\textwidth]{./figures/results/line_characterization_vi_vout25_vrot100_logtau7}
\includegraphics[height=0.28\textwidth]{./figures/results/line_characterization_vi_vout50_vrot100_logtau7}
\end{center}
\caption{\textbf{Valley Intensity:} We show for each \tauh the dependency that
		the viewing angle $\theta$ has on the line's the valley intensity. The larger
		$\cos{\theta}$ the lower is the valley intensity, so the higher $\theta$, the
		higher the valley intensity. Also, the higher \tauh, the better is the fit RT-DS.
		We fixed $\vrot=100$\kms.
		\label{fig:valley_intensity}}
\end{figure*}


\label{sec:results}

\subsection{Qualitative Trends}
\label{sec:qualitative}
Figure \ref{fig:doppler_shift} summarizes the most important trends.
The six panels correspond to $\tau=10^6$ and a viewing angle of
$\theta =90^{\circ}$, that is, perpendicular to the rotation axis of the
galaxy. 
In every panel the thin black line corresponds to the pure outflow
solution, i.e. without rotation. 
From top to bottom we see the effect of increasing the outflow
velocity, which is the expected increasing asymmetry towards the red
peak. 

The thick black line corresponds to the solution that includes both
outflows and rotation.
Comparing the left and right columns (lower versus higher rotational
velocity) we can see two immediate effects.
First, the line broadens and second, the intensity at the line's
center increases.

The thick gray line correspods to the pure outflow solution
with the Doppler boost added to model rotation's influence.
At $\tauh=10^6$ the Doppler boost does a good job at capturing the broad
morphological features introduced by rotation: the angle dependence,
the broadening and the intensity increase at the line's center.


In Figure \ref{fig:doppler_shift_theta_0} we show the same results as
in Figure \ref{fig:doppler_shift} but for a viewing angle of $\theta =
90^{\circ}$, that is parallel to the rotation axis. 
In this case we confirmethe result presented by \cite{Garavito14},
namely that pure rotation introduces a strong dependence with 
viewing angle, a trend that we find also holds for rotation mixed with
ouflows.   

The quality of the results from the Doppler boost improves for higher
\tauh values. 
In the Appendix we show the same plots as Figures
\ref{fig:doppler_shift} and \ref{fig:doppler_shift}, there it is
evident that for $\tauh=10^5$ the results are not as good as they are
for $\tauh=10^6$, and that for $\tauh=10^7$ the Doppler boos
provides a remarkable good approximation.

\subsection{Quantitative trends}
\label{sec:quantitative}


After finding the quantitative influence of the different parameters we
move onto a qualitative study.
To do this we summarize the line morphology by four different
characteristics: standard deviation (STD), skewness (SKW), bimodality
(BI) and valley/peak ration.
These quantities are defined by the following equations \citep{kokoska1999}.:

\begin{equation}
\label{eq:std}
\mathrm{STD} = \sqrt{m_2},
\end{equation}

\begin{equation}
\label{eq:skw}
\mathrm{SKW} = \frac{m_3}{m_2^{3/2}},
\end{equation}

\begin{equation}
\label{eq:bi}
\mathrm{BI} = \mathrm{KURTOSIS} - \mathrm{SKW}^2 = \frac{m_4}{m_2^{2}} - \frac{m_3^2}{m_2^{3}},
\end{equation}
%
where each $m_i$ is the i-th moment about the mean. 
The STD has velocity units and quantifies the line's width.
The SKW is adimensional and quantifies the peaks' asymmetry. 
In the case of a bimodal distribution, $SKW>0$ means that the blue
peak is taller and for $SWK<0$ the red peak is taller. 
The BI is adimensional and quantifies whether the line has 1 or 2
peaks: it is  always $\geq 1$ \citep{Pearson1929} and the 
to 1, the more bimodal is the line (i.e. has 2 similar peaks). 
We found by visual inspection of our spectra that BI$=2.5$ marks the
transition between two peaks (however imbalanced) and a dominang
single peak.


\subsubsection{Standard Deviation}
Figure \ref{fig:standard_deviation} summarizes the standard deviation
results for all our models.
Each panel shows the STD as a function of \vrot.
All panels were computed using a viewing angle of $\theta =
90^{\circ}$ (perpendicular to the rotation axis), which has the most
extreme ifluence from rotation.
 The black triangles
correspond to the full RT solution and the line to the DS
approximation.  
The optical depth increases from top to bottom and the outflow
velocity from left to right.
This quantitative plot confirms that the line width increases with
rotational velocity and  optical depth.
The DS successfuly reproduces all trends with the optical depth,
rotational velocity and outflow velocity.
However, the DS consistently underestimates the STD. 
The difference between the RT and DS increases with the outflow
velocity and the rotational velocity, and decreases with increasing
optical depth.
In the range of parameter space explore, this difference has as an
upper bound of $\sim 7\%$, $3\%$ and $\sim 2\%$ for
 $\tauh=10^5$, $10^6$ and $10^7$, respectively. 

\subsubsection{Skewness}

Figure \ref{fig:skewness} presents the skwewness results for all the
models together with the DS comparison following the same layout as
Figure \ref{fig:standard_deviation}.
In all cases the skewness is negative showing that all the lines
are unbalanced towards the red side of the spectrum.
Skewness increases with rotational velocity and decreases with
optical depth; rotation tries to smoothes the line diminishing the
asymetries while a higher optical depth reinforces the line asymetries.
The skewness does not have a monotonous trend with outflow velocity because
there is a transition between double and single peak line; for low
outflow velocities the skewness signals the balance between the two
existing peaks while for high outflow velocities it quantifies the
asymmetry of the already dominant read peak.
The DS reproduces the main trends, again with an underestimation that
decreases at higher optical depths and increases with larger values of
the rotational velocity and outflow velocity.
In this case the differences between RT and DS have an upper bound of
$85\%$, $35\%$ and $5\%$ 
for  $\tauh=10^5$, $10^6$ and $10^7$,
respectively.  


\subsection{Bimodality}
Figure \ref{fig:bimodality} shows the results for the bimodality using
the same layout as in the two previous Figures.
Following the reasoning about the skewness, we observe that
increasing the outflow velocity increases the value of bimodality,
that is, it transitions to a more pronounced single peak. 
The trend as a function of the rotational velocity and the optical
depth are not monotonous.
When the outflow velocity is low (\vout$<50\ \kms$), an increasing
rotational velocity smears the two asymmetrical peaks pushing the line
morphology towards a single peaks, making the bimodality statistics
increase. 
On other situations (\vout$=50\ \kms$ and \tauh$\geq 10^6$) higher
rotational velocities the bimodality statistics decreases, which means
that it manages to slightly enlarge the already dominant red peak.
The DS continous to reproduce the main trends, still underestimating
the bimodality statistics.
As expected from the previous results the difference betwen RT and DS
decreases at higher optical depths and increases with increasing
values of the rotational and outflow velocities.
In this case the differences have an upper bound of $4\%$, $2\%$ and
$1\%$ for  $\tauh=10^5$, $10^6$ and $10^7$, respectively.  


\section{Discussion}
\label{sec:discussion}

\subsection{Theoretical Insights}
\color{red}
Why Doppler Shift works. Talk about radiative transfer...
\color{black}
In Fig. \ref{fig:doppler} we present the spectrum of a LAE taken from to different
sides of the galaxy. As the LAE is rotating, one side is being redshifted while
the other is blueshifted. We see that the combined spectrum is a weighted line,
in solid black, that combines these past two. This shows that there is in fact a
Doppler Shift and then it justifies our choice to apply it after the outflows spectrum is
obtained.

\begin{figure}
  \begin{center}
    \includegraphics[width=0.48\textwidth]{./figures/discussion/doppler}
  \end{center}
  \caption{\textbf{Doppler Shift:} We fix $\vout=25$\kms, $\vrot=100$\kms, $\tauh=10^5$.
    \label{fig:doppler}}
\end{figure}

A first approach to an analytical expression that returns the \lya spectrum
from a rotating galaxy is presented by \cite{Garavito14}. This derivation is
based on the assumption that the distribution of photons' propagation directions
at the edge of the galaxy is anisotropic and depends of \tauh only. So this
approximation becomes more accurate with higher optical depth and it is also the
reason why the Doppler Shift technique to induce rotational effects to an outflow
\lya spectrum works.

On the other hand, regarding the parameters \vrot, \vout, and $\theta$ for
which this model would work, we have the following. The effect of the outflows
velocity is much more sensitive on the spectrum than the rotational one; meaning
that for a high \vout, the \vrot value would have to be equal or greater in order
to be able to glimpse the shift in frequency that rotation induces. Regarding $\theta$,
the introduction of rotation in the model implies that it will only affect galaxies
with its viewing angle $\theta \neq 90^\circ$. This can be observed in the cartesian
components of the velocity of the atoms (Eq. \ref{eq:vx}, \ref{eq:vy}, \ref{eq:vz})
where the only velocity factor affecting $v_z$ is the outflows one.

\subsection{Application to observational data}

We find three different approaches how this model can have observational
implications. Firstly, as the resolution of astronomical instruments is
increasing, new images from LAEs can be spatially resolved in a way that
rotation velocities of the galaxy can be determined. New spectrographs like
MUSE should help obtain kinematic information from LAEs by using our model.
As for now, previous works in the literatures have already found useful data
to support this idea. Fig 7 of Prescott et al. \cite{Prescott14} shows the
presence of Doppler shift and \cite{Herenz2016} creates 2D line of sight
velocity maps of several LARS (Lyman Alpha Reference Sample) galaxies.

Another application of this new model to the current interpretation of \lya
spectra, is that the central ($V=0$) emission of the spectra seen in the
\lya line, is a consequence of the viewing angle of the galaxy and can be
controlled by it. Several authors have suggested that this central emission
is caused by radiation that escapes the galaxy without scattering. So we
present an alternative that solves this issue. In addition, the effect of
outflows velocities \vrot in current models can modify the asymmetry of peaks
present in the \lya line. However the intensity of the valley and the width
of the line were not easily reproduced. In this work we provide a tool to
provide these effects with values of \vout and \vrot that do not exceed the
typical LAEs' values found in the literature.

\color{red}
Kulas...
\color{black}
Finally, we present a reconstruction of a selected \lya spectrum presented
by \cite{Kulas12} and reproduced with the permission of the author. We create
a fit of the \lya line and predict its kinematic properties.

\section{Conclusions}
\label{sec:conclusions}

In this paper we explore, for the first time in the literature,
the results of a model for the emergent \lya\ line from rotating outflows.
The main results for the model are computed from a Monte-Carlo radiative transfer
simulation and confronted with a simple semi-analytic ansatz that adds the effects
of rotation onto results of pure outflow kinematics.

The main effects of rotation on the \lya\ line morphology are:


\begin{itemize}
  \item Broadening the line.
	\item Increasing the intensity at the line's center.
	\item Inducing a dependency on the viewing angle,
  The closer the observer is to the pole of the galaxy (defined by the rotation axis),
  the smaller are the effects from rotation.
\end{itemize}

We found that in the case of solid body rotation these effects can be quantitatively
explained by a Doppler boost, where the Doppler factor can be computed as the
 producet of quantities at the surface of last scattering, namely $\vec{v}_{\rm rot}\cdot \hat{k}$,
 where $\vec{v}_{\rm rot}$ is the velocity due to rotation and $\hat{k}$ is the direction
of the photon's propagation.
These conclusions, specially the confirmation of the Doppler boost as
a good proxy for solid body rotation, strenghten the evidence reported
by \cite{Garavito14} when they modeled the effect of pure rotation on
\lya\ spectra.  

As an application to observational data we find that recent results
that take the spectra of two different sides of a galaxy can detect 
the effect of approaching/receeding gas motions. 
However, the distances between the peaks of these two spectra 
correspond to $\approx v_{\rm rot} \cos\theta$ due to the weights of the
emitting regions.




\bibliographystyle{mnras}
\bibliography{references}


\appendix

\section{Additional figures}
\label{sec:appendix}


\begin{figure}
  \begin{center}
    \includegraphics[width=0.49\textwidth]{./figures/results/doppler_shift_logtau5_theta90}
  \end{center}
  \caption{\textbf{Qualitative trends of changing outflow and
      rotational velocity.}
    Same layout as Figure \ref{fig:doppler_shift},
    this time  $\tauh=10^5$ and $\theta=90^\circ$.}
\end{figure}


\begin{figure}
  \begin{center}
    \includegraphics[width=0.49\textwidth]{./figures/results/doppler_shift_logtau7_theta90}
  \end{center}
  \caption{\textbf{Qualitative trends of changing outflow and
      rotational velocity.}
    Same layout as Figure \ref{fig:doppler_shift},
    this time  $\tauh=10^7$ and $\theta=90^\circ$.}
\end{figure}



\end{document}
