\documentclass[11pt,a4paper]{article}
\usepackage{graphicx}
\usepackage{amssymb}
\usepackage[latin1]{inputenc}    %% european characters can be used (Windows, old Linux)
\usepackage[T1]{fontenc}   %% get hyphenation and accented letters right

\pagestyle{empty} % no page numbers
\usepackage[left=35mm, right=35mm, top=15mm, bottom=20mm, noheadfoot]{geometry}

\begin{document}
\thispagestyle{empty}

\title{\textbf{Influencia de rotaci\'on y outflows en la\\
			   l\'inea espectral Lyman Alpha}}
		
\author{Autora:\\ \vspace{2mm}
		Maria Camila Remolina-Gutierrez - mc.remolina197@uniandes.edu.co\\ 
		Director: \\ \vspace{3mm}
		Jaime E. Forero-Romero - je.forero@uniandes.edu.co\\ 
		Departamento de F\'{i}sica, Universidad de los Andes, Bogot\'a, Colombia}
\date{} % <--- leave date empty
\maketitle\thispagestyle{empty} %% <-- you need this for the first page
\hyphenation{emi-si\'on ellas es-tu-dia-ron}

%Introducci�n y motivaci�n
Las galaxias son la clave fundamental para entender el Universo.  Cuando 
observamos una galaxia distante estamos viendo el pasado. A partir de este esfuerzo
de arqueolog\'ia astron\'omica podemos reconstruir la historia del Universo.  

Este proceso de reconstrucci\'on empez\'o en el Siglo XX. Los astr\'onomos 
estudiaron en detalle las galaxias m\'as cercanas, haciendo observaciones de
un Universo ya maduro. A finales del Siglo XX y comienzos del XXI, se logran 
obtener se\~nales de galaxias mucho m\'as distantes; conjuntos de gas y estrellas 
que emitieron su luz cuando el Universo era joven y ten\'ia menos de un quinto 
de su edad actual. Estas galaxias son tan lejanas, que las im\'agenes aportan 
poco al momento de deducir las propiedades de una galaxia. En estos casos, mayor 
parte de la informaci\'on viene en los espectros, es decir, en la descomposici\'on 
de la luz en radiaci\'on de diferentes longitudes de onda. De esta manera, el reto 
astrof\'isico se convierte en poder darle a estos espectros de galaxias distantes
una  interpretaci\'on.   

En los primeros esfuerzos conjuntos de interpretaci\'on y observaci\'on, se hizo
claro que varias galaxias j\'ovenes deben tener una alta cantidad de hidr\'ogeno 
y estrellas en proceso de formaci\'on. Bajo estas condiciones, gran parte de su luz
tiene la longitud de onda de la l\'inea de emisi\'on Lyman-$\alpha$, correspondiente
a la transici\'on de niveles de energ\'ia m\'as probable del \'atomo de hidr\'ogeno.
Las galaxias que fueron observadas con esta fuerte l\'inea de emisi\'on se denominaron
Lyman Alpha Emitters (LAEs). La deducci\'on de las propiedades f\'isicas de una 
galaxia a partir de esta l\'inea de emisi\'on se considera, en gran parte, como un
problema abierto. Esto debido a que su complejidad requiere de t\'ecnicas 
computacionales para ser atacado. 

%Breve estado del arte
En esta tesis utilizo t\'ecnicas de transferencia radiativa computacional masivamente 
paralela para interpretar espectros de LAEs. Propongo considerar a las LAEs como una 
distribuci\'on esf\'erica de \'atomos de Hidr\'ogeno que tienen un movimiento de rotaci\'on
superpuesto a movimiento radial de gas proyectado (outflow). Esto se motiva en que la 
rotaci\'on es un fen\'omeno com\'un a todas las galaxias y en que los outflows son una 
consecuencia natural de la explosi\'on de supernovas en una galaxia. Este es un tipo 
de modelo que no hab\'ia sido considerado, hasta la fecha, para la interpretaci\'on de 
espectros de LAEs. 

%Resultados principales y conclusi�n corta
Encuentro que la rotaci\'on sumada a una cantidad modesta de outflow, es suficiente 
para reproducir  las caracter\'isticas principales de LAEs observadas. Esto se encuentra
en contraste con la perspectiva m\'as aceptada en la comunidad, que espera velocidades
alt\'isimas de outflow para reproducir las observaciones, algo que desde el punto de 
vista energ\'etico es dificil de justificar para una galaxia joven. Estos resultados 
dan pie para revisar las interpretaciones que se han hecho hasta ahora sobre el estado 
f\'isico de galaxias j\'ovenes; haciendo un aporte a nuestro conocimiento para una 
reconstrucci\'on m\'as acertada de la historia de la evoluci\'on del Universo. 

%------------------------REFERENCES----------------------------

\bibliographystyle{apj}
\bibliography{references}

\end{document}
%%https://www.modelica.org/events/modelica2014/authors-guide/example-abstract.tex/view
