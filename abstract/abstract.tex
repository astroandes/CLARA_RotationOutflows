%\affil{Departamento de F\'{i}sica, Universidad de los Andes, Cra. 1 No. 18A-10, Edificio Ip, Bogot\'a, Colombia}
%\email{mc.remolina197@uniandes.edu.co}
%\email{je.forero@uniandes.edu.co}

\documentclass[11pt,a4paper]{article}
\usepackage{graphicx}
\usepackage{amssymb}
\usepackage[latin1]{inputenc}    %% european characters can be used (Windows, old Linux)
\usepackage[T1]{fontenc}   %% get hyphenation and accented letters right

% do not change these lines:
\pagestyle{empty}                %% no page numbers!
\usepackage[left=35mm, right=35mm, top=15mm, bottom=20mm, noheadfoot]{geometry}

\begin{document}
\thispagestyle{empty}

\title{\textbf{Influence of galaxy rotation and outflows \\
			   in the Lyman Alpha spectral line}}
		
\author{Maria Camila Remolina-Gutierrez$^1$, Jaime E. Forero-Romero$^1$\\ \vspace{3mm}
	    mc.remolina197@uniandes.edu.co, \hspace{0.8mm} je.forero@uniandes.edu.co\\ 
		$^1$ Departamento de F\'{i}sica, Universidad de los Andes \\
		Cra. 1 No. 18A-10, Edificio Ip, Bogot\'a, Colombia}
\date{} % <--- leave date empty
\maketitle\thispagestyle{empty} %% <-- you need this for the first page
\textit{Keywords - Galaxies: high-redshift, Lyman Alpha Emission, Galaxy Rotation, Galaxy Outflows, Radiative Transfer.}\\

Young galaxies in the Universe have a strong Ly-$\alpha$ emission caused by the ionized Hydrogen atoms in their interstellar medium. When the spectrum of a galaxy has an intense peak around the Ly-$\alpha$ natural frequency ($2.46\times 10^{15}$ Hz) it is called a Lyman Alpha Emitter (LAE). Typical LAEs are very distant ($z \gtrsim 2$). As a result all the data astronomers can obtain from them is their spectra, and from there all the physical information of the galaxy must be derived. Trying to solve this task requires the creation of a simplified and solid model. In this work we propose to consider LAEs as a spherical distribution of Hydrogen atoms that undergoes a solid body rotation and a radial expansion due to outflows. \\

We simulate the effect of rotational velocity ($v_{\mathrm{rot}}$), outflow velocity ($v_{\mathrm{out}}$) and optical depth ($\tau_{\mathrm{H}}$) of a LAE on its outgoing spectrum. This spectrum is obtained by considering the photons' path through the galaxy as a radiative transfer process. Then we used CLARA (Code for Lyman Alpha Radiation Analysis), a radiative transfer code that returns the resulting Ly-$\alpha$ line from the combination of the 3 parameters $v_{\mathrm{rot}}$, $v_{\mathrm{out}}$ and $\tau_{\mathrm{H}}$.\\

The main conclusion is that this new model reproduces LAEs observed features in a clear way and with consistent physical parameters. We find that rotation and a modest outflow quantity is enough to reproduce the primary characteristics of observed LAEs. This result goes against the most accepted community's perspective, which consists of having really high outflows in order to reproduce observations. However for such young galaxies these high outflows are energetically really hard to explain. We show that the outflows could be lowered by adding galaxy rotation to the model. The resulting spectra are roughly coherent with LAEs' observations and all the effects the parameters cause in the Ly-$\alpha$ profile, have been previously obtained by different authors. Nonetheless, proper observational fits are left for future work. This work accomplishes the objective of extracting as much information as possible from a LAE's Ly-$\alpha$ line.\\

%------------------------REFERENCES----------------------------

\bibliographystyle{apj}
\bibliography{references}

\end{document}