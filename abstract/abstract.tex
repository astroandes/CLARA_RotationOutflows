%\affil{Departamento de F\'{i}sica, Universidad de los Andes, Cra. 1 No. 18A-10, Edificio Ip, Bogot\'a, Colombia}
%\email{mc.remolina197@uniandes.edu.co}
%\email{je.forero@uniandes.edu.co}

\documentclass[11pt,a4paper]{article}
\usepackage{graphicx}
\usepackage{amssymb}
\usepackage[latin1]{inputenc}    %% european characters can be used (Windows, old Linux)
\usepackage[T1]{fontenc}   %% get hyphenation and accented letters right

% do not change these lines:
\pagestyle{empty}                %% no page numbers!
\usepackage[left=35mm, right=35mm, top=15mm, bottom=20mm, noheadfoot]{geometry}

\begin{document}
\thispagestyle{empty}

\title{\textbf{More rotation and less outflows can explain Lyman-Alpha observed line features}}
		
\author{Maria Camila Remolina-Gutierrez$^1$, Jaime E. Forero-Romero$^1$\\ \vspace{3mm}
	    mc.remolina197@uniandes.edu.co, \hspace{0.8mm} je.forero@uniandes.edu.co\\ 
		$^1$ Departamento de F\'{i}sica, Universidad de los Andes \\
		Cra. 1 No. 18A-10, Edificio Ip, Bogot\'a, Colombia}
\date{} % <--- leave date empty
\maketitle\thispagestyle{empty} %% <-- you need this for the first page
\textit{Keywords - Galaxies: high-redshift, Lyman Alpha Emission, Galaxy Rotation, Galaxy Outflows, Radiative Transfer.}\\

Radiative transfer MonteCarlo simulations are nowadays a required tool
to interpret the Lyman-$\alpha$ line morphology. In this work we
explore a new model that computes the joint effect of outflows and
bulk rotation. The main conclusion is that rotation and a modest
outflow velocity is enough to reproduce the primary characteristics of
observed LAEs. 
We will show results of adjusting observational data for some selected
objects to models with and without bulk rotation. We finalize by
discussing the possible implications for these results in terms of the
energetics required for supernova feedback and outflows in high
redshift galaxies. 


%------------------------REFERENCES----------------------------

\bibliographystyle{apj}
\bibliography{references}

\end{document}
